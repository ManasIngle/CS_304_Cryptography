
\documentclass[11pt]{article}
\usepackage[hmargin=1in,vmargin=1in]{geometry}
\usepackage{xcolor}
\usepackage{mdframed}
\newmdenv[linecolor=blue,backgroundcolor=blue!10]{mybox}
\usepackage{tikz}
\usepackage{enumitem}
\usepackage{amsmath,amssymb,amsfonts,url,sectsty,framed,tcolorbox,framed}
\newcommand{\pf}{{\bf Proof: }}
\newtheorem{theorem}{Theorem}
\newtheorem{lemma}{Lemma}
\newtheorem{proposition}{Proposition}
\newtheorem{definition}{Definition}
\newtheorem{remark}{Remark}
\newcommand{\qed}{\hfill \rule{2mm}{2mm}}
\usepackage{enumitem}



\begin{document}
%%%%%%%%%%%%%%%%%%%%%%%%%%%%%%%%%%%%%%%%%%%%%%%%%%%%%%%%%%%%%%%%%%%%%
\noindent
\rule{\textwidth}{1pt}
\begin{center}
{\bf [CS304] Introduction to Cryptography and Network Security}
\end{center}
Course Instructor: Dr. Dibyendu Roy \hfill Winter 2023-2024\\
Scribed by: Manas Jitendrakumar Ingle (202151086) \hfill Lecture 1,2 (Week 4)
\\
\rule{\textwidth}{1pt}
%%%%%%%%%%%%%%%%%%%%%%%%%%%%%%%%%%%%%%%%%%%%%%%%%%%%%%%%%%%
%write here

\section{Subgroup}

A non-empty subset $H$ in a group $(G, *)$ is a subgroup of $G$ if $H$ is itself a group with respect to the operation $*$ of $G$. If it is a proper subset and a group with respect to $*$ of $G$ and $H \neq G$, then $H$ is called a proper subgroup of $(G, *)$.

\begin{enumerate}
    \item $H \subseteq G$ or not
    \item $H$ is itself a group with $*$
\end{enumerate}

\textbf{Property:}

$(G, *)$ is a Group

$a \in G \rightarrow a * a \in G, a * a * a \in G$

$a * a = a^2, a * a * a = a^3 \in G$

$a^i = a * a * \ldots * a \in G, (a * a)^i \rightarrow i$ operations $*$

\section{Generators and Cyclic Group}
Consider a group $(G, *)$. Let $\alpha \in G$. The identity element $\alpha^0$ belongs to G. Therefore,
\begin{center}
    $\alpha^0 * \alpha = \alpha^1$\\
    \vspace{1mm}
    $\alpha^1 * \alpha = \alpha^2$\\
    \vspace{1mm}
    $\alpha^2 * \alpha = \alpha^3$\\
\end{center}
\textbf{Note: }The $*$ here is not multiplication, it is a binary operation not necessarily multiplication. $\alpha^1, \alpha^2, \alpha^3$ and so on, are just notation of using the binary operation $*$ on same element. \\
\vspace{5mm}
Since, G is closed under $*$, any two elements belonging to G, will give the result in G on performing the binary operation $*$. Since $\alpha^0 \in G$ and $\alpha \in G$, therefore $\alpha^1 \in G$. Now, since $\alpha^1 \in G$, therefore $\alpha^2 \in G$, and so on. That means,
\begin{center}
    $\alpha^0, \alpha^1, \alpha^2, ... \in G$
\end{center}
The set $\alpha^0, \alpha^1, \alpha^2, ...$ is denoted by $\langle \alpha \rangle$. Also, $\langle \alpha \rangle \subseteq G$. $\alpha$ is called the generator of $(G, *)$ iff:
\begin{center}
    for any $b \in G$ $\exists$ $i \geq 0$ such that $b = \alpha^i$ and hence $G \subseteq \langle \alpha \rangle $. 
\end{center}
\textbf{We can conclude that $(G,*) = \langle \alpha \rangle $}\\
A group is called a cyclic group if there is an element $\alpha \in G$, such that for every $b \in G$, there is an integer $i$ with $b = \alpha^i$. In simple words, every element in G can be expressed as some exponent of $\alpha$, then $\alpha$ is the generator of G.
\vspace{3mm}
\subsubsection{Order of an element}
Consider (G,*) and $|G|$ : finite. Let a$\in$G.\\
We already know that $a^0$ is identity. Now, the order of an element is the least positive integer m such that $a^m = e$. 
\begin{center}
    o(a) = m such that $a^m$ = e
\end{center}
Since $a^m = e$, so $a^{m+1}=a$, $a^{m+2}=a^2$ and so on. So we define a set H such as:
\begin{center}
    H = $\{a^0, a^1, a^2.....a^{m-1}\}$
\end{center}
We understand that
\begin{itemize}
    \item $H \subseteq G$
    \item H is a group under *
\end{itemize}


\section*{Lagrange's Theorem:}
If G is a finite group and H is a subgroup of g then $|H|$ divides $|G|.$
\begin{itemize}
    \item G is a finite group \\
    $a \in G$ \\
    $O(a) \mid |G|$ \\
    $\Rightarrow a \in G $\\\\ 
    $H = \{\ e = a^0, a^1, a^2, \dots, a^{O(a)-1} \}\ $ \\
    H is a subgroup of G \\\\
    From Lagrange's theorem: \\
    $|H| \mid |G| $\\
    $\Rightarrow O(a) \mid |G|$
    \item If the order of $a \in G $ is t \\
    then $O(a^k) = \frac{t}{gcd(t, k)}$
    \item If $gcd(t,k) = 1$\\
    then $O(a^k) = t = O(a)$\\
    $\Rightarrow  |<a^k>| = |<a>| $ \\\\
    $x \in <a^k>$\\
    $\Rightarrow x = (a^k)^i = a^{ki} = <a>$ \\
    
        $<a^k> \subseteq <a>$ \\
        $<a^k> = <a>$ \\
        $<a^k> \subseteq <a>$ \\
        $\Rightarrow <a^k> = <a>$\\
        {$a^k$ is also a generator of $<a>$}\\
    $<a^k> = <a> $ Subgroup generated by a \\
    $<a> = <a^k>$ Subgroup generated by $a^k$ 
\end{itemize}

\section{Ring:}
\subsection{Introduction:}
A ring $(R, +_R, \times_R)$ consists of one set R with two binary operations arbitrarily denoted by $+_R$(addition) and $\times_R$(multiplication) on R, satisfying the following properties: 
\begin{enumerate}
    \item $(R, +_R)$ is an abelian group with the identity element $0_R$
    \item The operation $\times_R$ is associative,i.e, \\
    $a \times_R (b \times_R c) $
    \item There is a multiplication identity denoted by $1_R$ with  $1_R \ne 0_R$ such that\\
    \begin{itemize}
        \item $1_R \times_R a = a \times_R 1_R = a \hspace{1cm} \forall a \in R$
    \end{itemize}
    \item The operation $\times_R$ is distributive over $+_n$ ,i.e,\\
    $(b +_R c) \times_R a = (b \times_R a) +_R (c \times_R a)$ \\
    $a \times_R (b +_R c) = (a \times_R b) +_R (a \times_R c)$ \\
\end{enumerate}






\section*{Field}
A field is a non-empty set F together with two binary operation +(addition) and *(multiplication) fow which the following properties are satisfied
\begin{itemize}
    \item (F, +) is an abelian group
    \item If $0_F$ denotes the additive identity element of (F,+) then (F $\backslash \{0_F\}, *) $ is a commutative/abelian group.
    \item $\forall$ a,b,c $\in$ F, we have,
    \begin{center}
        a*(b+c) = (a*b) + (a*c)
    \end{center}
\end{itemize}

\textbf{Note:}\\
\begin{itemize}
    \item $(Z, +, \cdot)$ is not a field because inverse does not exist
    \item $(Q, +, \cdot)$\\
    (Q, +) : abelian group\\
    0 : additive identity\\
    1 : multiplicative identity\\
    $(Q \backslash \{0\}, \cdot$ forms an abelian group.\\
    Hence, it is a field.
\end{itemize}
\textbf{Example:} Is $(\mathbb{F}_p, +_p, *_p)$ a field, where p is a prime number?\\
\textbf{Solution:} We know that $(\mathbb{F}_p, +_p)$ an abelian group with identity element 0. Now, the set $\mathbb{F}_p - \{0\}$ has existing multiplicative inverse iff gcd(x, p) = 1 for each $x \in \mathbb{F}_p - \{0\}$. Since, p is prime, gcd(x, p) = 1 for all possible integers that $x$ can take. Hence, $(\mathbb{F}_p, +_p, *_p)$ is a field.\\


%Lecture 9 

\section{Field Extension}
Suppose $K_2$ is a field with addition(+) and multiplication(*). \\
Suppose $K_1  K_2$ is closed under both these operations such that $K_1$ itself is a field with the restriction of + and * to the set $K_1$. Then $K_1$ is called a subfield of $K_2$ and $K_2$ is called a field extension of $K_1$.

As $K_1$ is a subset of $K_2$. Let $F$ be a field $(F, +, *)$. Consider the polynomial ring $F[x]$, which consists of all polynomials with coefficients in the field $F$:
\[ F[x] = \{a_0 + a_1x + a_2x^2 + \ldots | a_i \in F\} \]
The addition operation of two polynomials in $F[x]$:
\[
(a_0 + a_1x + a_2x^2 + \ldots + a_{n-1}x^{n-1}) + (b_0 + b_1x + b_2x^2 + \ldots + b_{n-1}x^{n-1})
\]
results in:
\[
(a_0 + b_0) + (a_1 + b_1)x + (a_2 + b_2)x^2 + \ldots + (a_{n-1} + b_{n-1})x^{n-1}
\]
where $a_i + b_i$ is the additive operation in the field $F$. The multiplication operation of two polynomials in $F[x]$:
\[
(a_0 + a_1x + a_2x^2 + \ldots + a_{n-1}x^{n-1}) * (b_0 + b_1x + b_2x^2 + \ldots + b_{n-1}x^{n-1})
\]
results in:
\[
(a_0b_0) + (a_0b_1 + a_1b_0)x + \ldots + (a_{n-1}b_{n-1})x^{2n-2}
\]

\section{Irreducible Polynomial}
A polynomial $P(x) \in F[x]$ of degree $n \geq 1$ is called irreducible if it cannot be written in the form of $P_1(x) * P_2(x)$ with $P_1(x), P_2(x) \in F[x]$ and degree of $P_1(x), P_2(x)$ must be greater than or equal to 1. It means that $P(x)$ is irreducible if it can not be factorised.\\
\newline
\textbf{Example:} $x^2 + 1 \in \mathbb{F}_2[x]$.\\
\textbf{Solution:} $(x + 1) * (x + 1) = x^2 + (1 + 1) \cdot x + 1 = x^2 + 1$. Therefore, $(x^2 + 1) =  (x + 1) * (x + 1)$ in $\mathbb{F}_2[x]$. Hence, $(x^2 + 1)$ is reducible in $\mathbb{F}_2[x]$. Note that it is not possible to factor $x^2 + 1$ in $\mathbb{R}[x]$, where $\mathbb{R}$ is set of real numbers.\\


\section{Advanced Encryption Standard:}
\begin{itemize}
    \item It is Standardized by NIST.
    \item Rijndael - winner of Advanced Encryption Standard Competition.
    \item Winner of the Competition was named AES.
\end{itemize}
AES is based on -
\begin{enumerate}
    \item Iterative block cipher.
    \item It is based on SPN.
\end{enumerate}
\subsection{Types of AES:}
\begin{enumerate}
    \item AES - 128
    \begin{enumerate}
        \item Block size = 128 bit
        \item Number of Rounds = 10
        \item Secret key size = 128 bit
    \end{enumerate}
    \item AES - 192
    \begin{enumerate}
        \item Block size = 128 bit
        \item Number of Rounds = 12
        \item Secret key size = 192 bit
    \end{enumerate}
    \item AES - 256
    \begin{enumerate}
        \item Block size = 128 bit
        \item Number of Rounds = 14
        \item Secret key size = 256 bit
    \end{enumerate}
\end{enumerate}

\end{document}